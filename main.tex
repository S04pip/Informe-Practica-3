\documentclass{article}
\usepackage[utf8]{inputenc}
\usepackage[spanish]{babel}
\usepackage{enumitem}
\usepackage{geometry} 
\usepackage{array}
\usepackage{graphicx}
\usepackage{booktabs}
\usepackage{tikz}
\usetikzlibrary{shapes, arrows}

\geometry{
    a4paper,
    left=2.5cm,
    right=2.5cm,
    top=2.5cm,
    bottom=2.5cm
}

\title{Informe Practica 3}
\author{Samuel Primera C.I: 31.129.684, Samuel Reyna CI: 30.210.759}
\date{29 de Julio 2025}

\begin{document}

\maketitle

\section{Explique cómo se organiza la memoria cuando un sistema utiliza memory-mapped I/O. ¿En qué región de memoria se suelen mapear los dispositivos? ¿Qué implicaciones tiene para las instrucciones lw y sw?}

En un sistema con \textit{memory-mapped I/O}, los dispositivos de entrada/salida (E/S) se mapean en el espacio de direcciones de memoria principal. Esto significa que las direcciones de memoria reservadas para E/S se asignan a registros de control, estado o datos de los dispositivos, en lugar de a celdas de memoria RAM. Los dispositivos suelen mapearse en regiones específicas del espacio de direcciones, como las direcciones altas o reservadas por el sistema.


Para las instrucciones \texttt{lw} (load word) y \texttt{sw} (store word), esto implica que pueden utilizarse para interactuar con los dispositivos de E/S como si fueran memoria. Por ejemplo, al leer (\texttt{lw}) una dirección mapeada a un registro de estado, se obtiene el estado del dispositivo, y al escribir (\texttt{sw}) en una dirección mapeada a un registro de control, se envía un comando al dispositivo. No se requieren instrucciones especiales para E/S, lo que simplifica el diseño del procesador.

\section{¿Cuál es la principal diferencia entre memory-mapped I/O y la entrada/salida por puertos? ¿Qué ventajas y desventajas tiene cada enfoque? ¿Por qué MIPS32 utiliza principalmente memory-mapped I/O?}

\begin{itemize}
    \item \textbf{Diferencia}:
    \begin{itemize}
        \item \textit{Memory-mapped I/O} integra los dispositivos en el espacio de direcciones de memoria, usando instrucciones de carga/almacenamiento (\texttt{lw/sw}) para E/S.
        \item \textit{E/S por puertos} utiliza un espacio de direcciones separado y requiere instrucciones específicas (como \texttt{in} y \texttt{out} en x86) para acceder a los dispositivos.
    \end{itemize}
    
    
\subsection*{E/S Mapeada en Memoria}

\textbf{Ventajas:}
\begin{itemize}
    \item Se pueden utilizar las mismas instrucciones de acceso a memoria para acceder a los registros de E/S.
    \item No se requieren instrucciones especiales de E/S.
    \item Se puede utilizar cualquier tipo de direccionamiento disponible para la memoria.
    \item Se pueden mapear tantos registros como direcciones de memoria haya disponibles.
\end{itemize}

\textbf{Desventajas:}
\begin{itemize}
    \item Se reduce el espacio de direcciones disponible para la memoria principal.
    \item Se requiere un mecanismo para proteger los registros de E/S del acceso no autorizado por parte de programas de usuario (normalmente gestionado por el sistema operativo).
\end{itemize}

\subsection*{E/S No Mapeada en Memoria (E/S Aislada)}

\textbf{Ventajas:}
\begin{itemize}
    \item No se reduce el espacio de direcciones de la memoria principal.
    \item Es más fácil controlar el acceso a los registros de E/S, ya que se requieren instrucciones específicas.
\end{itemize}

\textbf{Desventajas:}
\begin{itemize}
    \item Se necesitan instrucciones especiales de E/S (como IN y OUT), lo que puede complicar la lógica del procesador.
    \item El número de registros de E/S que se pueden direccionar suele ser limitado.
    \item Se pueden requerir señales de control adicionales en el bus del sistema para distinguir entre accesos a memoria y accesos a E/S.
\end{itemize}

    
    MIPS32 utiliza principalmente \textit{memory-mapped I/O} porque su arquitectura RISC favorece la simplicidad del conjunto de instrucciones. Al no necesitar instrucciones especiales para E/S, se reduce la complejidad del hardware y se mantiene un diseño más limpio y eficiente.

\section{En un sistema con memory-mapped I/O: ¿Qué problemas pueden surgir si dos dispositivos usan direcciones solapadas? ¿Cómo se evita este conflicto?}

En un sistema con \textbf{Memory-Mapped I/O}, los dispositivos de entrada/salida se comunican con el CPU a través de direcciones de memoria específicas. Si dos dispositivos comparten o se solapan en las mismas direcciones, pueden surgir los siguientes problemas:

\begin{itemize}[label=--]
    \item \textbf{Acceso erróneo a dispositivos}: El CPU podría leer o escribir en una dirección esperando interactuar con un dispositivo, pero en realidad estaría accediendo al otro.
    \item \textbf{Activación no deseada}: Algunos dispositivos pueden reaccionar incluso ante una lectura. Si comparten una dirección, ambos podrían activarse simultáneamente.
    \item \textbf{Corrupción de datos}: Es posible sobrescribir datos importantes si el CPU accede a una dirección compartida sin saberlo.
    \item \textbf{Lecturas peligrosas}: En Memory-Mapped I/O, incluso leer una dirección puede desencadenar una acción en el hardware.
\end{itemize}

\subsection*{Soluciones}

Para evitar estos conflictos, se aplican las siguientes estrategias, también válidas en arquitecturas como MIPS32:

\begin{itemize}[label=--]
    \item \textbf{Asignación única de direcciones}: Cada dispositivo debe tener una dirección única en el espacio de memoria.
    \item \textbf{Uso de instrucciones byte específicas}: Se utilizan instrucciones como \texttt{BITB} o \texttt{TSTB} para acceder a un solo byte, evitando tocar direcciones adyacentes.
    \item \textbf{Decodificación precisa en hardware}: El sistema de memoria incluye lógica que redirige correctamente las solicitudes del CPU a cada dispositivo.
    \item \textbf{Separación de registros}: Los registros de estado y datos de los dispositivos se asignan a direcciones distintas, evitando solapamientos.
\end{itemize}

\section{¿Por qué se considera que el memory-mapped I/O simplifica el diseño del conjunto de instrucciones de un procesador? ¿Qué tipo de instrucciones adicionales serían necesarias si se usara E/S por puertos?}

El \textit{memory-mapped I/O} utiliza las mismas instrucciones de carga/almacenamiento (\texttt{lw/sw}) para acceder tanto a memoria como a dispositivos de E/S. Esto elimina la necesidad de instrucciones especializadas para E/S.

En cambio, con E/S por puertos, el procesador requeriría instrucciones adicionales como:
 in para leer de un puerto.
 out para escribir en un puerto.
 Esto aumenta la complejidad del diseño del procesador y del compilador, ya que
 se deben manejar dos espacios de direcciones (memoria y puertos) con instrucciones distintas

\section{¿Qué ocurre a nivel del bus de datos y direcciones cuando el procesador accede a una dirección de memoria que corresponde a un dispositivo? ¿Cómo sabe el hardware que debe acceder a un periférico en lugar de la RAM?}

Cuando el procesador accede a una dirección mapeada a un dispositivo:
\begin{enumerate}
    \item La dirección se coloca en el bus de direcciones.
    \item El controlador de memoria o un circuito de decodificación determina, basándose en el rango de direcciones, si la solicitud corresponde a RAM o a un dispositivo.
    \item Si es un dispositivo, el controlador habilita el acceso al periférico correspondiente y bloquea el acceso a RAM para esa dirección.
    \item Los datos se transfieren a través del bus de datos entre el procesador y el dispositivo.
\end{enumerate}

El hardware sabe diferenciar entre RAM y periféricos gracias a la decodificación de direcciones, donde ciertos rangos están preasignados para E/S durante el diseño del sistema.

\section{¿Es posible que un programa normal (sin privilegios) acceda a un dispositivo mapeado en memoria? ¿Qué mecanismos de protección existen para evitar accesos no autorizados?}

No, un programa normal sin privilegios no puede acceder directamente a un dispositivo mapeado en memoria. Esto se debe a que los sistemas modernos implementan varios mecanismos de protección, tanto a nivel de hardware como de software, para garantizar la seguridad y estabilidad del sistema operativo.

\subsection*{¿Por que?}
\begin{itemize}
  \item Los dispositivos mapeados en memoria (como controladores, periféricos, etc.) están ubicados en \textbf{regiones especiales} de la memoria física que el \textbf{sistema operativo reserva} para el kernel.
  \item Un programa normal se ejecuta en \textbf{modo usuario}, y este modo tiene restricciones severas de acceso para proteger el sistema contra errores o acciones maliciosas.
  \item Intentar acceder a esas regiones sin permiso provoca una \textbf{excepción de protección} (por ejemplo, una falla de segmentación), lo que generalmente termina el programa.
\end{itemize}

\subsection*{Mecanismos que impiden el acceso}

\begin{center}
\begin{tabular}{>{\bfseries}m{4cm} m{10cm}}
\toprule
Mecanismo & Función principal \\
\midrule
Modo de ejecución & El procesador distingue entre modo \textit{usuario} y \textit{kernel}. Solo el kernel puede acceder directamente a memoria de dispositivos. \\
MMU (Memory Management Unit) & Traduce direcciones virtuales a físicas y controla los accesos a cada región. \\
Tablas de páginas & Determinan qué páginas pueden ser leídas, escritas o ejecutadas por cada proceso. \\
Syscalls y drivers & El acceso a dispositivos se realiza indirectamente mediante llamadas al sistema y controladores confiables. \\
Espacio de direcciones & Cada proceso tiene su propio entorno virtual; no puede ver ni modificar el del kernel o de otros procesos. \\
\bottomrule
\end{tabular}
\end{center}


\bigskip

Esto garantiza que el hardware no pueda ser manipulado directamente por procesos maliciosos o con errores. Así se protege la \textbf{integridad del sistema} y se evita que un programa pueda, por ejemplo, modificar el estado de un controlador de disco o leer datos confidenciales desde memoria compartida.

\section{¿Qué técnicas se pueden emplear para evitar esperas activas innecesarias al interactuar con dispositivos?}

Para evitar esperas activas como el \textit{sondeo} (\textit{polling}), que consiste en verificar repetidamente el estado de un dispositivo o registro de hardware para determinar si está listo para realizar una operación, que consumen recursos del CPU, se usan:
\begin{itemize}
    \item \textbf{Interrupciones}: El dispositivo notifica al procesador cuando está listo, liberando al CPU para otras tareas hasta que ocurra la interrupción.
    \item \textbf{DMA (Direct Memory Access)}: Permite a los dispositivos transferir datos directamente a memoria sin intervención del CPU.
    \item \textbf{Buffering}: Almacenar datos en búferes para que el CPU no deba esperar a que el dispositivo esté listo en cada operación.
\end{itemize}

Estas técnicas mejoran la eficiencia del sistema al reducir el tiempo de espera del CPU.

\section{Análisis y Discusión de los Resultados}

Aunque la finalidad sea distinta, ambos algoritmos son similares en cuanto a su funcionamiento. Inicializando registros específicos para marcar el correcto funcionamiento del programa.
En el sensor de temperatura, al inicializarse el sensor de estado y recibir el valor 1 se inicializa el sensor de control, lo que habilita la lectura de la temperatura y para luego imprimirla con su código. Similar al algoritmo anterior se inicializa el registro de control de tensión para iniciar la medición, se calcula la tensión sistólica y la tensión diastólica y se guardan en sus respectivos registros, y se concluye la medición guardando el valor 1 en el registro de estado.
Aunque se utilizan registros particulares, esto permite visualizar en la memoria de datos el estado de los procesos, además de facilitar la utilización de dispositivos de E/S mapeados en memoria.

\end{document}